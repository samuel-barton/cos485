\section*{Problem 1}

In this problem, we are asked to do out the N-queens problem by hand with N=9.
\\
\\
Below is a table containing our solution. 

For each queen we have assigned a color as follows:

\begin{figure}[hb]

\centering

\begin{minipage}{0.4\textwidth}
\centering
\begin{description}
	\setlength \itemsep{0.25pt}
	\item [queen 1] --- \textcolor{red}{red} 
	\item [queen 2] --- \textcolor{blue}{blue}
	\item [queen 3] --- \textcolor{green}{green}
	\item [queen 4] --- \textcolor{yellow}{yellow}
	\item [queen 5] --- \textcolor{cyan}{cyan}
	\item [queen 6] --- \textcolor{teal}{teal}
	\item [queen 7] --- \textcolor{magenta}{magenta}
	\item [queen 8] --- \textcolor{olive}{olive}
	\item [] 
\end{description}	
\end{minipage}
\hfill
\begin{minipage}{0.4\textwidth}
\centering
\begin{TAB}(e){|c|c|c|c|c|c|c|c|c|}{|c|c|c|c|c|c|c|c|c|}
	$\cdot$ & $\cdot$ & \textcolor{red}{$\mathrm{Q}_1$} & $\cdot$ & $\cdot$ & $\cdot$ & $\cdot$ & $\cdot$ & $\cdot$ \\
	$\cdot$ & $\textcolor{red}{\mathrm{x}_1}$ & $\textcolor{red}{\mathrm{x}_1}$ & $\textcolor{red}{\mathrm{x}_1}$ & $\cdot$ & $\cdot$ & $\cdot$ & $\cdot$ & \textcolor{blue}{$\mathrm{Q}_2$} \\
	$\textcolor{red}{\mathrm{x}_1}$ & \textcolor{green}{$\mathrm{Q}_3$} & $\textcolor{red}{\mathrm{x}_1}$ & $\cdot$ & $\textcolor{red}{\mathrm{x}_1}$ & $\cdot$ & $\cdot$ & $\textcolor{blue}{\mathrm{x}_2}$ & $\textcolor{blue}{\mathrm{x}_2}$ \\
	$\textcolor{green}{\mathrm{x}_3}$ & $\textcolor{green}{\mathrm{x}_3}$ & $\textcolor{red}{\mathrm{x}_1}$ & $\cdot$ & $\cdot$ & $\textcolor{red}{\mathrm{x}_1}$ & $\textcolor{blue}{\mathrm{x}_2}$ &  \textcolor{yellow}{$\mathrm{Q}_4$} & $\textcolor{blue}{\mathrm{x}_2}$ \\
	\textcolor{cyan}{$\mathrm{Q}_5$} & $\textcolor{green}{\mathrm{x}_3}$ & $\textcolor{red}{\mathrm{x}_1}$ & $\textcolor{green}{\mathrm{x}_3}$ & $\cdot$ & $\textcolor{blue}{\mathrm{x}_2}$ & $\textcolor{red}{\mathrm{x}_1}$ & $\textcolor{yellow}{\mathrm{x}_4}$ & $\textcolor{blue}{\mathrm{x}_2}$ \\
	$\textcolor{cyan}{\mathrm{x}_5}$ & $\textcolor{green}{\mathrm{x}_3}$ & $\textcolor{red}{\mathrm{x}_1}$ & \textcolor{teal}{$\mathrm{Q}_6$} & $\textcolor{blue}{\mathrm{x}_2}$ & $\textcolor{yellow}{\mathrm{x}_4}$ & $\cdot$ & $\textcolor{red}{\mathrm{x}_1}$ & $\textcolor{blue}{\mathrm{x}_2}$ \\
	$\textcolor{cyan}{\mathrm{x}_5}$ & $\textcolor{green}{\mathrm{x}_3}$ & $\textcolor{red}{\mathrm{x}_1}$ & $\textcolor{blue}{\mathrm{x}_2}$ & $\textcolor{yellow}{\mathrm{x}_4}$ & $\textcolor{green}{\mathrm{x}_3}$ & \textcolor{magenta}{$\mathrm{Q}_7$} & $\textcolor{yellow}{\mathrm{x}_4}$ & $\textcolor{red}{\mathrm{x}_1}$ \\
	$\textcolor{cyan}{\mathrm{x}_5}$ & $\textcolor{green}{\mathrm{x}_3}$ & $\textcolor{red}{\mathrm{x}_1}$ & $\textcolor{yellow}{\mathrm{x}_4}$ & \textcolor{olive}{$\mathrm{Q}_8$} & $\textcolor{teal}{\mathrm{x}_6}$ & $\textcolor{green}{\mathrm{x}_3}$ & $\textcolor{yellow}{\mathrm{x}_4}$ & $\textcolor{blue}{\mathrm{x}_2}$ \\
	$\textcolor{cyan}{\mathrm{x}_5}$ & $\textcolor{blue}{\mathrm{x}_2}$ & $\textcolor{red}{\mathrm{x}_1}$ & $\textcolor{teal}{\mathrm{x}_6}$ & $\textcolor{cyan}{\mathrm{x}_5}$ & $\textcolor{olive}{\mathrm{x}_8}$ & $\textcolor{teal}{\mathrm{x}_6}$ & $\textcolor{green}{\mathrm{x}_3}$ & $\textcolor{blue}{\mathrm{x}_2}$ \\
\end{TAB}
\end{minipage}

\end{figure}

In the above table we demonstrate the use of the Monte Carlo approach to generate an estimate of the total cost to solve the N-queens problem with N=9. In the below table we give the row sums for 
all of the options, and then the total sum which represents the overall cost to solve the problem:
\\
\\
\begin{centering}
\begin{tabular}{l | c | l}
	\textbf{Row \#} & \textbf{Choices} & \textbf{Cost} \\
	1 & 9 & 9 \\
	2 & 6 & $9 \cdot 6$ \\
	3 & 4 & $9 \cdot 6 \cdot 4$ \\
	4 & 3 & $9 \cdot 6 \cdot 4 \cdot 3$ \\
	5 & 2 & $9 \cdot 6 \cdot 4 \cdot 3 \cdot 2$ \\
	6 & 2 & $9 \cdot 6 \cdot 4 \cdot 3 \cdot 2 \cdot 2$ \\
	7 & 1 & $9 \cdot 6 \cdot 4 \cdot 3 \cdot 2 \cdot 2 \cdot 1$ \\
	8 & 1 & $9 \cdot 6 \cdot 4 \cdot 3 \cdot 2 \cdot 2 \cdot 1 \cdot 1$ \\
	9 & 0 & 0 \\
	\hline
	\textbf{Total} & & \textbf{9999}
\end{tabular}
\end{centering}