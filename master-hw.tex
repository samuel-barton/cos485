%--------------------------------------------------------------
% Template author: Ben Montgomery
% Current revision: 1/23/2017
%--------------------------------------------------------------

\documentclass[12pt]{article}
 
\usepackage[margin=1in]{geometry} 
\usepackage{amsmath, amsthm, listings, amssymb}
\usepackage[nottoc]{tocbibind}
\usepackage{float, mathtools}
\usepackage{array,multirow}
\usepackage{caption}
\usepackage{graphicx, tikz, }% Include figure files
\usepackage{rotating}
\usetikzlibrary{positioning}
\usetikzlibrary{decorations.markings}
\tikzset{main node/.style={circle,fill=blue!20,draw,minimum size=1cm,inner sep=0pt},}
\tikzset{min node/.style={circle,fill=blue!20,draw,minimum size=0.5cm,inner sep=0pt},}

 % ----------New special chars of computer scientists----------
\DeclarePairedDelimiter\ceil{\lceil}{\rceil}
\DeclarePairedDelimiter\floor{\lfloor}{\rfloor} 
\newcommand{\N}{\mathbb{N}}
\newcommand{\R}{\mathbb{R}}
\newcommand{\Z}{\mathbb{Z}}
 
 % --------------New problem-solving enviroments--------------
\newenvironment{exercise}[2][Exercise]{\begin{trivlist}
\item[\hskip \labelsep {\bfseries #1}\hskip \labelsep {\bfseries #2.}]}{\end{trivlist}}
\newenvironment{problem}[2][Problem]{\begin{trivlist}
\item[\hskip \labelsep {\bfseries #1}\hskip \labelsep {\bfseries #2.}]}{\end{trivlist}}

\newenvironment{solution}{\begin{proof}[Solution]}{\end{proof}}

% --------------------------------------------------------------
\begin{document}
\title{Homework 5 -- COS 485}
\author{Samuel Barton \and Ben Montgomery}
\date{April 3, 2017}
 
\maketitle
%---------------------------Problem---------------------------
\begin{problem}{1}
Use the Monte Carlo algorithm by hand (as done in class) to estimate the number
of promising nodes in a backtracking search for an N-Queens problem of size N = 9.
\end{problem}

\begin{problem}{2}
Use the backtracking algorithm for the sum-of-subsets problem to find a combination of the numbers [4, 6, 15, 17, 23, 26, 31] that sums to 50. You can stop after the first combination is found. Show your work in the form of a recursive tree diagram like Figure 5.9 showing each choice along the way.
\end{problem}

\begin{problem}{3}
Exercise 5.22 – Note: you are really creating a prototype of a graph that can be expanded to any desired number of vertices. 
\end{problem}

\begin{problem}{4}
Apply the greedy algorithm from the previous problem to color the graph on the right. Does it produce an optimal coloring?
\end{problem}

\begin{problem}{5}
Create a small graph for which the greedy algorithm does not find an optimal coloring
\end{problem}

\begin{problem}{6}
Exercise 5.22 – Note: you are really creating a prototype of a graph that can be expanded to any desired number of vertices.
\end{problem}
\begin{problem}{7}
Demonstrate the Backtracking algorithm for the Hamiltonian Circuits Problem (Algorithm 5.6) using the graph on the right. When there is a choice, visit the nodes in numerical order. Show the search tree of choices and stop at the first solution. Note: don’t draw anything in your search tree for edges that do not exist.
\end{problem}
\begin{problem}{8}
Design a branch and bound algorithm to solve the Sequence Alignment problem from Section 3.7. Base your algorithm on the possible
choices for dealing with the leading character of the two sequences. Do not write pseudocode. Explain what the state space vector of your search is, what the set of possibilities for each choice are, your bounding function, and what the worst-case execution time would be.
\end{problem}

\begin{problem}{9}
There are 4 people (a son, a father, a grandfather, and a great grandfather) trekking together in Nepal at night. They have come to a very long, old and rickety bridge that they must cross as quickly as possible. The bridge can only support the
weight of two people at a time, and furthermore they only have a single high-tech flashlight indicating a remaining battery life of 17 minutes. Any party that crosses the bridge, either 1 or 2 people, must have the flashlight with them. The 4 people walk at very different speeds: the sprightly young son can run across the bridge in one minute, the father takes two
minutes, the grandfather who uses a cane takes five minutes, and the great grandfather who uses a walker takes 10 minutes. If two people cross together they must walk at the speed of the slower person. How can they all cross the bridge to the other side in 17 minutes?
\end{problem}
\end{document}