\section*{Problem 5}

In this problem we are given the following situation and asked to find the 
minimun number of afternoons required to solve it.

\begin{quote}
    A list of 17 committees with 200 unique members where no member can be in more
    than 5 committees. Each committee meeting takes one afternoon to complete.
\end{quote}
%
This problem is most similar to the graph coloring problem. We cast this problem
into the graph coloring problem in the following way:
\\
\\
First, each job becomes a vertex in the graph. Each edge in the graph represents a
conflicting committee, namely a pair of committees which share a member. In order 
to determine whether a pair of committees conflict we take a linear pass through 
their respective membership lists and look for a common member.
\\
\\
With this graph built we simply run the graph coloring algorithm and return the 
minimun number of colors needed to color the graph. This number is the number of 
afternoons needed to hold all of the meetings with no conflicts.
\begin{center}
To demonstrate this algorithm we consider the following example: 
\\\hfill
\\
    \begin{tabular}{| c | c  c  c| c c | }
        \hline
        \textbf{Committee} & \multicolumn{3}{l|}{\textbf{Members}} & 
        \multicolumn{2}{l|}{\textbf{Conflicts With}} \\
        \hline
        W & A & C & & X & Y \\
        \hline
        X & C & D & & W & Z \\
        \hline
        Y & A & B &E& W & Z \\
        \hline
        Z & B & D & & X & Y \\
        \hline
    \end{tabular}
\end{center}
%
Where we note that we are given that the optimal solution takes two afternoons. 
Below we present the graph representing these committees and the conflicts between
them.
\\
\begin{center}
    \begin{tikzpicture}
       \node[red node] (1) {W};
       \node[green node] (2) [right = 2cm of 1]{X};
       \node[green node] (3) [below = 2cm of 2]{Y};
       \node[red node] (4) [below = 2cm of 1]{Z};

       \draw (1) -- (2) 
             (1) -- (3)
             (2) -- (4)
             (3) -- (4);
    \end{tikzpicture}
\end{center}
%
Note that this is a valid coloring for the graph, and that there are two colors
needed to color this graph. This means that, by our algorithm statement, we 
can schedule all of the meetings in two afternoons with no conflicting meetings.
