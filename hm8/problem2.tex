\section*{Problem 2}

We are given two problems $A$, $B$, such that $A$ can be converted to $B$ in 
$t_n \in \Theta(n^a)$, a polynomial quantity, meaning $A \propto B$ if the solution to $B$ has 
$t_n \in \Theta(n^b)$, a polynomial quantity. To bound the complexity of this 
solution, we will refer to the below proof:

\begin{proof}
Suppose we have an instance of problem $A$ that is of size $n$. Because at 
most there are $\Theta(n^a)$ steps in the transformation algorithm, and at 
worst the algorithm outputs a symbol at each step, the size of the instance 
of $B$ produced by the transformation is at most a quantity $s$ such that 
$s \in \Theta(n^a)$. When that instance is the input to the algorithm for $B$, 
this means there are at most $\Theta\left((n^a)^b\right)$ steps. Therefore, 
the maximum number of steps required to transform the instance of problem $A$ 
to an instance of problem $B$ and then solve problem $B$ to get the correct 
answer for problem $A$ is at most
$$
\Theta(n^a) + \Theta\left((n^a)^b\right)
$$
which is a polynomial in $n$.
\end{proof}
