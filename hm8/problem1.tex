\section*{Problem 1}

In this problem, we are asked to classify the problem of finding the $n$th Fibonacci term, given the following categories:
\begin{enumerate}
    \item Problems Solvable in Polynomial Time
    \item Problems Proven to be Intractable
    \item Problems Not Proven to be Intractable, but Lack Existing Polynomial Solutions
\end{enumerate}

As noted in section 1.3 of our text, when we use a binary representation for the input problem size, a standard representation used when evaluating if a problem can be solved in polynomial time. This means the input size is $\floor{\lg n} + 1$.

The algorithm for finding the Nth term in the Fibbonacci sequnce is linear time
in the magnitude of the nth term; however, by the rationale given in secion 9.2
the algorithm is exponential in the ``input size'' defined as the number of 
characters needed to store the input.

This algorithm cannot be intractable as it does not produce a nonpolynomial 
amount of output, or fall into the same category as the 
\textit{Halting Problem}. Thus we can assert that the problem is neither proven
to be intractable, but polynomial-time solutions have not been found.
