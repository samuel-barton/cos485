\section*{Problem 1}

In this problem, we are asked to classify the problem of finding the $n$th 
Fibonacci term, given the following categories:
\begin{enumerate}
    \item Problems Solvable in Polynomial Time
    \item Problems Proven to be Intractable
    \item Problems Not Proven to be Intractable, but Lack Existing Polynomial 
          Solutions
\end{enumerate}
%
In order to solve this problem we must define what the Fibonacci sequence is:

\begin{quote}
The Fibonacci sequence, which is discussed in Subsection 1.2.2 is 
defined as follows: 
%
\begin{align*}
t_n &= t_{n-1} + t_{n-2} \\
t_0 &= 0 \\
t_1 &= 1
\end{align*}
This definition gives us the recurrence equation for calculating the $n$th 
term in the Fibonacci sequence, regardless of implementation.
\end{quote}
%
In Appendix B, we are given the solution to this recurrence equation. It is 
presented below. 
%
$$
t_n = 
    \dfrac{\left[\dfrac{1 + \sqrt{5}}{2}\right]^n - 
           \left[\dfrac{1 - \sqrt{5}}{2}\right]^n}{\sqrt{5}}
$$
This is clearly not a polynomial. Also, this recurrence equation gives us a 
representation of the amount of output produced by this algorithm and not the
time complexity. 
\\
\\
Even though we subtract one exponential from another, as $n \rightarrow \infty$
the second term approaches zero as $\frac{1 - \sqrt{5}}{2} < 1$. Thus we
are left with a single exponential term. This means that the output produced
is non-polynomial in $n$.
\\
\\
The size of the input to this problem is not $n$, but rather is $2^d$ if we 
assume a binary representation scheme for numbers. $d$ is the number of binary
digits, which is defined by the text as the input size. Solving $n = 2^d$ for 
$d$ gives us $d = \floor{\lg(n)} + 1$. Thus our equation for the amount of 
output produced to represent the $n$th term in the Fibonacci sequence is 
$$
t_n = 
    \dfrac{\left[\dfrac{1 + \sqrt{5}}{2}\right]^{2^d} - 
           \left[\dfrac{1 - \sqrt{5}}{2}\right]^{2^d}}{\sqrt{5}}
$$
Thus the amount of output produced in representing the $n$th term in the 
Fibonacci sequence is non-polynomial with respect to $d$, where $d$ is the input 
size. Thus this problem is inherently intractable by definition as it produces
a non-polynomial amount of output.
