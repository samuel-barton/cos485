\section*{Problem 3}

It is stated that both problem $A$ and problem $B$ are decision problems such that $A \propto B$. It is also stated that $A$ is an NP-complete problem. We are then asked if $B$ is also $NP$-Complete.
\\
\\
Let us define $NP$-hard and $NP$-complete
\\
\begin{quote}
    A problem is $NP$-hard if there is an $NP$-complete problem which is 
    Turing reducable to it. If an $NP$-complete problem is polynomial-time
    many-one reducable to a problem, then it is also Turing reducable to that
    problem.
\end{quote}
\begin{quote}
    A problem is $NP$-complete if it is in the set $NP$ and is $NP$-hard.
\end{quote}
\begin{quote}
    \textit{Lemma}:

    If any $NP$-complete problem reduces to some problem, then every 
    $NP$-complete problem can be reduced to that problem.
\end{quote}
%
We consider the following two cases: 
\begin{enumerate}
    \item If $B$ is verifiable in polynomial time, then either $B \in P$ or 
          $B \in NP$, and since $P \in NP$ it must be true that $B \in NP$, 
          and so $B$ is $NP$-complete as $B \in NP$ and $B$ is $NP$-hard.
    \item If $B$ is not verifiable in polynomial time, then $B$ is intractable.
          We know that $B$ is $NP$-hard; however, we cannot determine if it is 
          also $NP$-complete as doing so would require solving the problem of 
          determining whether $P = NP$.
\end{enumerate}
%
Thus, we do not have enough information about $B$ to give a definitive answer
as to whether or not $B$ is $NP$-complete.
