\section*{Problem 3}

It is stated that both problem $A$ and problem $B$ are decision problems such that $A \propto B$. It is also stated that $A$ is an NP-complete problem. We are then asked if $B$ is also $NP$-Complete.
\\
Let us define NP-hard and Turing Reducable

\begin{quote}
    If a problem $A$ can be solved in polynomial time useing a hypothetical 
    polynomial-time algorithm for problem $B$, then problem $A$ is 
    polynomial-time Turing Reducable to problem $B$. We denote this
    $A \propto_T B$.
\end{quote}

\begin{quote}
    A problem $B$ is called $NP$-hard if, for some NP-complete problem $A$,
    $A \propto_T B$.
\end{quote}

Problem $B$ is either intractable, in $P$, or in $NP$. If $B$ is intractable, 
then $B$ is not in NP. If $B$ is in NP, then B is an $NP$-hard problem by 
definition. If, $B$ is in $P$, then $P = NP$ and $B$ is still an $NP$-hard
problem by definition.

Thus $B$ is an $NP$-hard problem.
