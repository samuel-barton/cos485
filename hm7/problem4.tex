\section*{Problem 4}

The flow of the graph from Problem 3 is maximal as domonstrated by the 
following minimum cut of forward paths. \textbf{A --- C}, 
\textbf{E --- C}, and
\textbf{E --- T} are all saturated forward paths. These paths span the
graph and thus form a minimum cut. The sum of the flow of these three
paths is 18, and the flow of the graph is now 18 as well. Below is the
final flow of the graph.

\begin{center}
    \begin{tikzpicture}
       \node[red node] (1) {S};
       \node[green node] (2) [above right = 2cm of 1]{A};
       \node[green node] (3) [below right = 2cm of 1]{D};
       \node[green node] (4) [above right = 2cm of 3]{C};
       \node[green node] (5) [above right = 2cm of 4]{B};
       \node[green node] (6) [below right = 2cm of 4]{E};
       \node[red node] (7) [below right = 2cm of 5]{T};

       \draw[largeptr]
       (1) -- (3) node[midway, below left]{10/12};
       \draw[largeptr]
       (1) -- (2) node[midway, above left]{3/9};
       \draw[largeptr]
       (2) -- (4) node[midway, above right]{7/7};
       \draw[largeptr]
       (4) -- (3) node[midway, above left]{0/4};
       \draw[largeptr]
       (3) -- (6) node[midway, below]{11/15};
       \draw[largeptr]
       (6) -- (4) node[midway, below left]{7/7};
       \draw[largeptr]
       (6) -- (7) node[midway, below right]{4/4};
       \draw[largeptr]
       (4) -- (5) node[midway, below right]{14/20};
       \draw[largeptr]
       (5) -- (7) node[midway, above right]{14/20};
       \draw[largeptr]
       (5) -- (2) node[midway, above]{0/4};
       
       \node[anchor=east] at (7.5,-2) (init) {};
       \node[anchor=east] at (2,0) (start) {};
       \node[] at (3.5,3) (end) {};
       \draw (start) edge[out=90,in=-90, densely dotted, ->, ultra thick] (end) edge[out=-90,in=90, densely dotted, ultra thick, ->] (init);
    \end{tikzpicture}
    \newline
    As demonstrated above, the minimum possible cut is $18$.
\end{center}
