\section*{Problem 1}

In this problem we categorize the following sorting algorithms into the
various genres of algorithms discussed in this course.
\\
\\
By definition, iterative improvement gradually improves an initial solution.
Thus it is impossible for a sorting algorithm to be an iterative improvement 
algorithm as a list is either sorted or unsorted. Once there is a solution,  no
improvements can be made on it.
\\
\\
\begin{tabular}{l c p{6cm}}
        \textbf{Algorithm} & \textbf{Technique} & \textbf{Justification} \\
        \hline
    Insertion sort & Greedy & Go through each element starting at position 0. Shift all of the elements greater than the current one to the left.\\
    Selection sort & Greedy & Find the smallest item in the unsorted part of the list and append it to the end of the sorted part. \\
    Bubble sort & Greedy & If the next value is smaller than the current one, swap them.\\
    Quicksort & Divide and Conquer & Divide the problem size recursively, sort at the bottom, then sort on your way up. Note that Quicksort \textit{can} be implemented with a randomized partition.\\
    Merge sort & Divide and Conquer & Divide the problem size recursively, sort at the bottom, then sort on your way up.\\
    Heap sort & Greedy & Larger values go higher (maxheap), or smaller values go higher (minheap)\\
\end{tabular}
