\section*{Problem 3}

In this problem we are to find a set of augmenting paths which will 
make the flow in Figure \ref{base-flow} maximal. 
\\
\\
We consider the following set of augmenting paths:
\begin{center}
    \begin{tabular}{ l | c }
    \textbf{Path} & \textbf{Flow Added} \\
    \hline
    \textcolor{red}{\textbf{S --- D --- E --- T}} & 3 \\
    \textcolor{blue}{\textbf{S --- A --- C --- B --- T}} & 6 \\
    \textcolor{cyan}{\textbf{S --- D --- E --- C --- B --- T}} & 6 \\
    \end{tabular}
\end{center}

These paths correspond to the following changes to the flow of the
graph.

\begin{center}
    \begin{tikzpicture}
       \node[red node] (1) {S};
       \node[green node] (2) [above right = 2cm of 1]{A};
       \node[green node] (3) [below right = 2cm of 1]{D};
       \node[green node] (4) [above right = 2cm of 3]{C};
       \node[green node] (5) [above right = 2cm of 4]{B};
       \node[green node] (6) [below right = 2cm of 4]{E};
       \node[red node] (7) [below right = 2cm of 5]{T};

       \draw[largeptr]
       (1) -- (3) node[midway, below left]{2/12};
       \draw[largeptr]
       (1) -- (2) node[midway, above left]{1/9};
       \draw[largeptr]
       (2) -- (4) node[midway, above right]{5/7};
       \draw[largeptr]
       (4) -- (3) node[midway, above left]{1/4};
       \draw[largeptr]
       (3) -- (6) node[midway, below]{3/15};
       \draw[largeptr]
       (6) -- (4) node[midway, below left]{2/7};
       \draw[largeptr]
       (6) -- (7) node[midway, below right]{1/4};
       \draw[largeptr]
       (4) -- (5) node[midway, below right]{6/20};
       \draw[largeptr]
       (5) -- (7) node[midway, above right]{2/20};
       \draw[largeptr]
       (5) -- (2) node[midway, above]{4/4};
       \draw[largeptr]
       (1) -- (3) node[red, midway, below left=15pt]{5/12};
       \draw[largeptr]
       (3) -- (6) node[red, midway, below=10pt]{6/15};
       \draw[largeptr]
       (6) -- (7) node[red, midway, below right=15pt]{4/4};
       \draw[largeptr]
       (1) -- (2) node[blue, midway, above left=15pt]{3/9};
       \draw[largeptr]
       (2) -- (4) node[blue, midway, above right=15pt]{7/7};
       \draw[largeptr]
       (4) -- (5) node[blue, midway, below right=15pt]{8/20};
       \draw[largeptr]
       (5) -- (2) node[blue, midway, above=10pt]{-4/4};
       \draw[largeptr]
       (5) -- (7) node[blue, midway, above right=15pt]{8/20};
       \draw[largeptr]
       (1) -- (3) node[cyan, midway, below left=30pt]{10/12};
       \draw[largeptr]
       (3) -- (6) node[cyan, midway, below=20pt]{11/15};
       \draw[largeptr]
       (6) -- (4) node[cyan, midway, above right]{7/7};
       \draw[largeptr]
       (4) -- (3) node[cyan, midway, above left=15pt]{-1/4};
       \draw[largeptr]
       (4) -- (5) node[cyan, midway, below right=30pt]{14/20};
       \draw[largeptr]
       (5) -- (7) node[cyan, midway, above right=30pt]{14/20};
    \end{tikzpicture}
\end{center}
%
The flow of this graph is maximal as domonstrated by the following 
minimum cut of forward paths. \textbf{A --- C}, \textbf{E --- C}, and
\textbf{E --- T} are all saturated forward paths. These paths span the
graph and thus form a minimum cut. The sum of the flow of these three
paths is 18, and the flow of the graph is now 18 as well. Below is the
final flow of the graph.

\begin{center}
    \begin{tikzpicture}
       \node[red node] (1) {S};
       \node[green node] (2) [above right = 2cm of 1]{A};
       \node[green node] (3) [below right = 2cm of 1]{D};
       \node[green node] (4) [above right = 2cm of 3]{C};
       \node[green node] (5) [above right = 2cm of 4]{B};
       \node[green node] (6) [below right = 2cm of 4]{E};
       \node[red node] (7) [below right = 2cm of 5]{T};

       \draw[largeptr]
       (1) -- (3) node[midway, below left]{10/12};
       \draw[largeptr]
       (1) -- (2) node[midway, above left]{3/9};
       \draw[largeptr]
       (2) -- (4) node[midway, above right]{7/7};
       \draw[largeptr]
       (4) -- (3) node[midway, above left]{0/4};
       \draw[largeptr]
       (3) -- (6) node[midway, below]{11/15};
       \draw[largeptr]
       (6) -- (4) node[midway, below left]{7/7};
       \draw[largeptr]
       (6) -- (7) node[midway, below right]{4/4};
       \draw[largeptr]
       (4) -- (5) node[midway, below right]{14/20};
       \draw[largeptr]
       (5) -- (7) node[midway, above right]{14/20};
       \draw[largeptr]
       (5) -- (2) node[midway, above]{0/4};
    \end{tikzpicture}
\end{center}
