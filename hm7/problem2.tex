\section*{Problem 2}
In this problem we are asked to explain why this flow is valid, although
not a maximum flow.
\begin{figure}[h]
    \centering
    \begin{tikzpicture}
       \node[red node] (1) {S};
       \node[green node] (2) [above right = 2cm of 1]{A};
       \node[green node] (3) [below right = 2cm of 1]{D};
       \node[green node] (4) [above right = 2cm of 3]{C};
       \node[green node] (5) [above right = 2cm of 4]{B};
       \node[green node] (6) [below right = 2cm of 4]{E};
       \node[red node] (7) [below right = 2cm of 5]{T};
    

       \draw[largeptr]
       (1) -- (3) node[midway, below left]{2/12};
       \draw[largeptr]
       (1) -- (2) node[midway, above left]{1/9};
        \draw[largeptr]
       (2) -- (4) node[midway, above right]{5/7};
       \draw[largeptr]
       (4) -- (3) node[midway, above left]{1/4};
       \draw[largeptr]
       (3) -- (6) node[midway, below]{3/15};
       \draw[largeptr]
       (6) -- (4) node[midway, below left]{2/7};
       \draw[largeptr]
       (6) -- (7) node[midway, below right]{1/4};
       \draw[largeptr]
       (4) -- (5) node[midway, below right]{6/20};
       \draw[largeptr]
       (5) -- (7) node[midway, above right]{2/20};
       \draw[largeptr]
       (5) -- (2) node[midway, above]{4/4};
    \end{tikzpicture}
    \caption{Flow graph with non-maximal flow}
    \label{base-flow}
\end{figure}
\begin{flushleft}
This flow is valid because the sum of the flow(s) into each node, besides the 
source and sink nodes, match the sum of the flow(s) out;
however this flow is not optimal as we have not saturated any of the forward
edges, and there is a backwards edge, namely \textbf{A --- B}, which 
has been saturated even though it should not have any flow whatsoever.
\end{flushleft}
