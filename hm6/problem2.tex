\section*{Problem 2}

In this problem we consider the case of one miserly king who has 
demanded a weighing of $N$ coins with the understanding that one coin
is counterfeit and lighter than the others. 
\\
\\
The algorithm to efficiently solve this is as follows:
\begin{enumerate}[itemsep=0.25em]
    \item Divide coins into three parts $A$, $B$, and $C$ where the size
          each is $\floor[\big]{\frac{n}{3}}$. If $N$ is not divisible
          by 3, then we will put the 1 or two additional coins in $D$
          and set them aside.
    \item Weigh two of $A$, $B$, or $C$.
    \subitem If $A = B$,  then throw out $A$ and $B$ and keep $C$
    \subitem If $A < B$, then throw out $B$ and $C$ and keep $A$
    \subitem If $B < A$, then throw out $A$ and $C$ and keep $B$
    \item Repeat until only one of $A$, $B$, and $C$ remain
    \item If $N$ is divisible by 3, then we are done. Otherwise, we
          have either 1 or 2 elements in $D$ to consider. Compare 
          whichever of $A$, $B$, or $C$ remains with 1 of the elements
          in $D$ using the above logic.
\end{enumerate} 
%
In the worst case, N is not divisible by 3 and has a remainder of 2. 
Thus there are 2 elements in $D$ and we do $\ceil[\big]{\log_3(n)}$ 
comparisons to end up with 1 element from $A$, $B$, or $C$ and two 
elements from $D$ to consider. We make one additional comparision with 
1 element from $D$ and the remaining element from the prior comparisons,
and can then determine which coin is the counterfeit. Thus in the worst
case our algorithm will have a time complexity equal to the number of
comparisons which is
$$
    T_n \; = \; \floor[\big]{\log_3(n)} + 1 \; = \; 
    \ceil[\big]{\log_3(n)} 
$$
