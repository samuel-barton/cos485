\section*{Problem 6}

In this continuation of the \textit{Miserly King} problem we are asked 
to 
determine the maximum number of coins we may weigh in three weighings 
if we add the condition that the counterfeit may be lighter or heavier
than the real coin. Also, we have access to an arbitrarily large number
of good coins to use as reference. We'll call the good pile $G$.
\\
\\
The algorithm to solve this is as follows:

\begin{enumerate}[noitemsep]
    \item Divide the coins up into three partitions $A$, $B$, and $C$ 
          as before
    \item weigh $A$ and $B$
    \item If $weight(A) = weight(B)$ then the counterfeit is in $C$
    \item If $weight(A) < weight(B)$ or $weight(A) > weight(B)$, 
             then weigh $A$ against $G$ and note whether $A$ was 
             heavier or lighter than $B$
    \item if $weight(A) < weight(G)$, then the counterfeit coin
                is in $A$, and it is lighter than the good coins
    \item if $weight(A) >  weight(G)$, then the counterfeit coin
                is in $A$, and it is heavier than the good coins
    \item if $weight(A) = weight(G)$, then the counterfeit coin is                in $B$ and we know its weight as we noted whether it 
                was heavier or lighter than $B$
    \item Now that we know whether the counterfeit is heavier or 
          lighter than the good coins we can continue as we did in the 
          original \textit{Miserly King} problem.
\end{enumerate}
%
Thus, while before we could weigh $N$ coins in $\ceil{\log_3 N}$ steps,
now that we have to do one more weighing to determine the weight 
difference between the counterfeit and a good coin, it will take  
$\ceil{\log_3 N} + 1$ weighings to weigh $N$ coins.
\\
\\
Solving this equation for $N$ with 3 weighings gives us the following:
$$
\ceil{\log_3 N} + 1 = 3 \implies 
\ceil{\log_3 N} = 2 \implies 
N = 3^2 = 9
$$
