\section*{Problem 1}

In this problem we are asked to make a decision tree argument about the
lower limit on the worst case number of comparisons needed to find a
key in a sorted list.
\\
\\
Let us assume our list has $N$ items. We assume that these items are in
sorted order. There are $N$ possible solutions, namely keys, from which 
we must choose one. Thus the height of the decision tree must be large
enough to cover the search space of $N$ items. Since we have sorted the 
items we have a binary decision, either an item is greater than its 
predecessor or it is less than its predecessor. Thus we have a binary 
tree where each node can have at most two children, the height of the 
tree can be calculated by solving the following equation.
\\
$$
    N \leq 2^h \; \implies \; 
    \ceil{\lg (N)} \leq h \; \implies \; 
    h \geq \ceil{\lg (N)}
$$
\\
Since the height of the decision tree is equal to the number of needed
comparisions, we have that there can be a minimum of $\lg N$
comparisions in the worst case to find a particular key in a sorted
list of $N$ elements.