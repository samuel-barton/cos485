%--------------------------------------------------------------
% Template author: Ben Montgomery
% Current revision: 1/23/2017
%--------------------------------------------------------------

\documentclass[12pt]{article}
\usepackage{amsmath} 
\usepackage[margin=1in]{geometry} 
\usepackage{amsthm, listings, amssymb}
\usepackage{float, mathtools}
\usepackage{breqn}
\usepackage{array,multirow,graphicx}

 % ----------New special chars of computer scientists----------
\DeclarePairedDelimiter\ceil{\lceil}{\rceil}
\DeclarePairedDelimiter\floor{\lfloor}{\rfloor} 
\newcommand{\N}{\mathbb{N}}
\newcommand{\R}{\mathbb{R}}
\newcommand{\Z}{\mathbb{Z}}
 
 % --------------New problem-solving enviroments--------------
\newenvironment{exercise}[2][Exercise]{\begin{trivlist}
\item[\hskip \labelsep {\bfseries #1}\hskip \labelsep {\bfseries #2.}]}{\end{trivlist}}
\newenvironment{problem}[2][Problem]{\begin{trivlist}
\item[\hskip \labelsep {\bfseries #1}\hskip \labelsep {\bfseries #2.}]}{\end{trivlist}}

\newenvironment{solution}{\begin{proof}[Solution]}{\end{proof}}

% --------------------------------------------------------------
\begin{document}
\title{Program 1 --- COS 485}
\author{Samuel Barton \and Ben Montgomery \and Tyler Nelson}
\date{May 1, 2017}
 
\maketitle
\section{Algorithm Overview}
This ``algorithm'' is, sadly, not an algorithm but a heuristic used to get close estimates to best-known solutions for the Steiner Tree problem. It works on the principal of using Dijkstra's shortest paths algorithm

\section{Algorithm Analysis}
Let us define a system of costs and variables as the following:
\begin{enumerate}
    \item Let us assign a cost of $1$ to basic mathematical operations, individual Boolean statements, if statements, array accesses, and variable assignments.
    \item Let datastructure (Arrays, HashMaps, Priority Queues...) initialization have a cost equal to the number of elements it is initially given the capacity for.
    \item Let us have $t$ target nodes in the Steiner tree.
    \item Let us have $v$ vertices in the graph.
    \item Let us have $e$ edges in the graph.
    \item Let $\Omega$ be Chitlin's Constant, the p
    \item Let a selected shortest path from Dijkstra's Shortest Paths algorithm have a length of $d$.
\end{enumerate}

\textbf{steinerTree:}
\begin{flalign*}
    1 + pathFinder + 1
\end{flalign*}

\textbf{pathFinder (Base Solution):}
\begin{flalign*}
    2 + t + 2 + 4e + v + 4v + v(getShortestPath + 5d) + 1 + 6v + 1 + 6e
\end{flalign*}

\textbf{pathFinder (Iterative Improvement Setup):}
\begin{flalign*}
    2v + 4 + updateNumberOfConnections + 2 + 4e + 2e + 1 + 4e + 6e 
\end{flalign*}

We must pause to recognize that in theory, the iterative improvement could run for an 
exceedingly long amount of time. The probability of this happening is incredibly slim,
and can be demonstrated by the Monte Carlo method; however, it is worth recognizing 
that in the worst case, this program could spend its time improving its solution by
some number bounded by $\lambda = e^e$.
\\
\textbf{pathFinder (Iterative Improvement):}
\begin{dmath*}
    \lambda 
        \left(
            2 +
            (v - t) \\
            \left( 
                5 + 
                3(v - 1) + 
                3 +
                (t - 1) \\
                \left(
                    1 + 
                    getShortestPath +
                    1 +
                    getPath +
                    3 (v - 1) 
                \right) \\+
                2 +
                7e +
                2 +
                v\\ +
                \left(
                    2 + 
                    4(v - 1) +
                    3
                \right) \\ + 
                v (2 + 1 + betterTrimmer) +
                3 +
                \left(
                    2 + v \\ +
                    (v-1)(4) 
                \right)
            \right) 
        \right) + 6e
\end{dmath*}

\textbf{getShortestPath:}
\begin{dmath*}
    1 + shortestPaths + 1 + 
    (6v + (t-1))
    + n \lg n + 1\\
    + (t-1)(7 + hasCycle + replacePath) + 1 + n \lg n + 2
\end{dmath*}

\textbf{getPath:}
\begin{dmath*}
   
\end{dmath*}
\end{document}
































